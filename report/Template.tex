% Template for ICIP-2019 paper; to be used with:
%          spconf.sty  - ICASSP/ICIP LaTeX style file, and
%          IEEEbib.bst - IEEE bibliography style file.
% --------------------------------------------------------------------------
\documentclass{article}
\usepackage{spconf,amsmath,graphicx}
\usepackage[backend=biber, style=ieee]{biblatex}
\addbibresource{reportRefs.bib}

% Example definitions.
% --------------------
\def\x{{\mathbf x}}
\def\L{{\cal L}}

% Title.
% ------
\title{House Price And Affordability In Regions Of Englanf And Wales}
%
% Single address.
% ---------------
% \name{Author(s) Name(s)\thanks{Thanks to XYZ agency for funding.}}

\name{Junsong Yang}
\address{4274056 \\
psyjy3@nottingham.ac.uk}

\begin{document}
%\ninept
%
\maketitle
%

\section{Introduction}
House price has changes dramatically over the past few years. Such changes are drawing 
attentions as they are leading to housing affordability issues. For example, house price 
in London increased drastically these years, as a result, house price is becoming unaffordable
to most of the residents. Analysis of how the house price changed and 
to what extend the affordability is affected by that changes is provided.

The dataset, obtained from office for national statistics, contains annual data from 1997 to 2018 about 
the median house price across regions of England and Wales, the median gross annual 
earnings based on working place associated with different regions and the ratio of 
median house price to median fross annual earnings.\cite{henretty_2019} \cite{henretty_data_2019}

\section{Research Questiions}
As mentioned above, the increase of house price lead affordability concerns. A better understanding of 
the house market, and affordability of the locals are necessary to study this issue. For instance, 
how the house price changed and to what extend that change has influenced affordability. Hence 
the research questions are proposed as follow.

\begin{enumerate}
  \item How the mean house price across the regions changed from 2000 onward? \\
        Which region has cheapest house?
\end{enumerate}


\subsection{Subheadings}


Subheadings should appear in lower case (initial word capitalized) in
boldface.  They should start at the left margin on a separate line.
 
\subsubsection{Sub-subheadings}


Sub-subheadings, as in this paragraph, are discouraged. However, if you
must use them, they should appear in lower case (initial word
capitalized) and start at the left margin on a separate line, with paragraph
text beginning on the following line.  They should be in italics.


\section{PAGE NUMBERING}


Please do {\bf not} paginate your paper.  Page numbers, session numbers, and
conference identification will be inserted when the paper is included in the
proceedings.

\section{ILLUSTRATIONS, GRAPHS, AND PHOTOGRAPHS}


Illustrations must appear within the designated margins.  They may span the two
columns.  If possible, position illustrations at the top of columns, rather
than in the middle or at the bottom.  Caption and number every illustration.
All halftone illustrations must be clear black and white prints.  Colors may be
used, but they should be selected so as to be readable when printed on a
black-only printer.

Since there are many ways, often incompatible, of including images (e.g., with
experimental results) in a LaTeX document, below is an example of how to do
this.

% Below is an example of how to insert images. Delete the ``\vspace'' line,
% uncomment the preceding line ``\centerline...'' and replace ``imageX.ps''
% with a suitable PostScript file name.
% -------------------------------------------------------------------------
\begin{figure}[htb]

\begin{minipage}[b]{1.0\linewidth}
  \centering
  \centerline{\includegraphics[width=8.5cm]{image1}}
%  \vspace{2.0cm}
  \centerline{(a) Result 1}\medskip
\end{minipage}
%
\begin{minipage}[b]{.48\linewidth}
  \centering
  \centerline{\includegraphics[width=4.0cm]{image3}}
%  \vspace{1.5cm}
  \centerline{(b) Results 3}\medskip
\end{minipage}
\hfill
\begin{minipage}[b]{0.48\linewidth}
  \centering
  \centerline{\includegraphics[width=4.0cm]{image4}}
%  \vspace{1.5cm}
  \centerline{(c) Result 4}\medskip
\end{minipage}
%
\caption{Example of placing a figure with experimental results.}
\label{fig:res}
%
\end{figure}


% To start a new column (but not a new page) and help balance the last-page
% column length use \vfill\pagebreak.
% -------------------------------------------------------------------------
%\vfill
%\pagebreak

\vfill\pagebreak
\printbibliography


\end{document}
