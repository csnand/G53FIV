% Template for ICIP-2019 paper; to be used with:
%          spconf.sty  - ICASSP/ICIP LaTeX style file, and
%          IEEEbib.bst - IEEE bibliography style file.
% --------------------------------------------------------------------------
\documentclass{article}
\usepackage{spconf,amsmath,graphicx}
\usepackage[backend=biber, style=ieee]{biblatex}
\addbibresource{reportRefs.bib}

% Example definitions.
% --------------------
\def\x{{\mathbf x}}
\def\L{{\cal L}}

% Title.
% ------
\title{House Price And Affordability In Regions Of Englanf And Wales}
%
% Single address.
% ---------------
% \name{Author(s) Name(s)\thanks{Thanks to XYZ agency for funding.}}

\name{Junsong Yang}
\address{4274056 \\
psyjy3@nottingham.ac.uk}

\begin{document}
%\ninept
%
\maketitle
%

\section{Introduction}
House price has changes dramatically over the past few years. Such changes are drawing 
attentions as they are leading to housing affordability issues. For example, house price 
in London increased drastically these years, as a result, house price is becoming unaffordable
to most of the residents. Analysis of how the house price changed and 
to what extend the affordability is affected by that changes is provided.

This paper is intend to explore this issue based data about house prices and affordability measurements. 
In general, there are four sections, Initial Question, Data Processing, Information Visualisation and Evaluation.
In Initial Questiions section, issuses related to house prices will be discussed therefore research questions 
will be proposed. As for Data Processing part, information about dataset used in this paper will be briefly 
introduced and how it was processed and data cleaning and transformation process will be explained. 
Visualisation strategies will be discussed in Information Visualisation section alongside visual encoding. 
As for the Evaluation part, justifications of choice of visual encodings and stragies will be 
provided and also the general reflection of development process.

\section{Initial Questiions}
As mentioned above, the increase of house price lead affordability concerns. A better understanding of 
the house market, and affordability of the locals are necessary to study this issue. For instance, 
how the house price changed and to what extend that change has influenced affordability. Hence 
the research questions are proposed as follow.

\begin{enumerate}
  \item How the mean house price across the regions changed 
  from 2000 onward and Which region has cheapest house.
  \item How the ratio of house price to workplace-based earnings
  changed from 2000 onward and which region has the most affordable
  houses.
  \item How the house price and affordability was affected by the economy crisis in 2008.
\end{enumerate}

% structures of visualisation part (two graphs at least)
% 1 describe graphs 150 -200
% 2 visualisation strategies 150 - 200
% 3 how those graphs anwsered the question 150 -200
% 4 (optional) further question emerged ? 100 -200



\section{Data Processing}

The dataset, obtained from office for national statistics, contains annual data from 1997 to 2018 about 
the median house price across regions of England and Wales, the median gross annual 
earnings based on working place associated with different regions and the ratio of 
median house price to median gross annual earnings.\cite{henretty_2019} \cite{henretty_data_2019}
In each part, only annual data is provided. 

The data is grouped by regions, counties and local authorities. Data of median house price, median 
affordability ratio and lower quartile house price is provided for each group. The affordability ratio 
refers to the ratio of house price to annual gross earnings. In this case the Affordability ratio was 
calculated as the ratio of median house price to annual gross workplace-based earnings. 
(work-based earnings refers the earnings based on where a person work and does not necessarily reveal 
the earnings of the local residents.)

Data cleaning and filtering process is essential for the project as the dataset has quite a few garbage data. 
As those initial questions suggest, data from 2000 to 2018 will be preserved and analysed, therefore, data 
out of this rage will be purged. As the data provided is completed so the missing data problem does not need 
to deal with. Since the data is also consistant, the entity resolution will not be a problem. 

Type conversion is an issue spacifically related to the R programming language. The original data came in 
xls form with the correct type for each category of data. But when loading those data into R, the default 
data type was character. This issue may cause severe problem whrm performing numeric analysis and visualisation. 
Therefore, data of house price, affordability ratio and data indicating time need to be converted into numeric type. 

As the the annual data is presented seperately in different columns, transformation is needed for further 
data analysis. The original data was put seperately in columns by each year, which would be diffcult to 
analyse changes based on timeline. Hence, the data was trnasformed into three-column form. Name of region, 
year, and the actual values of price (or earnings, or ratios) were in three individual column.

\section{Information Visualisation}
In this section, all three initial question proposed earlier will be discussed with visualisation. 
For each question, visualisation stragies and visual encoding will be explained in detail and 
criticial discussions of visualisation design will be included in this section. After the three initial 
questions were explored, exploratory process of proposing new question and visualisation of that question 
will be explored.

\subsection{House Price Visualisation}
As mentioned earlier, the first question is intend to examine how the house price changed from 2000 to 2018 
and to further probe which region has the lowest house price. As data processing operations described above. 
The first graph can be obtained.

\begin{figure}[htb]
  \begin{minipage}[b]{1.0\linewidth}
    \centering
    \centerline{\includegraphics[width=8.5cm]{Q1Geom_line}}
  %  \vspace{2.0cm}
    \centerline{Question 1: Result 1}\medskip
  \end{minipage}
\end{figure}

This line graph illustrate the median house prices in nine regions of England and Wales from 2000 to 2018. 
England and Wales are also treated as regions and also England and Wales combined. Therefore, there 12 lines 
in the graph that represent 12 regions individually. 

The overall tendency of change for all regions is quite similar. Although flucuated for a few years, 
the median house price in 12 regions was all increased from 2000 to 2018. Staring from 2000, median house 
price in most region rose steadily until 2014. From 2004 to 2008, for all the regions, the increase of 
median house price was slowing down comparing to the increase from 2000 to 2004. For the first time from 2000, 
median house price in all regions suddenly dropped to the level of two years before. Then in 2009, the median 
house price in all regions bounced back to the highest level from 2000. From 2010 onward, the median house 
price was almost fixed for 3 years. Starting from 2013, the median hose price started increasing steadily for 
most regions but London. Median house price in London experienced a sharp increase from 2013 to 2017, 
as the result, the median house price increased by 1/3 compare to the data in 2013. From 2017 onward, 
the increase was slowing down again.

This line graph as an example is expressive and efficient when representing time-series data or the changing 
of quantitative during a continuous period of time. But sometimes, it is not appropriate for conducting 
comparison. Therefore, the second graph was ontained using the same median house price data.

\begin{figure}[htb]
  \begin{minipage}[b]{1.0\linewidth}
    \centering
    \centerline{\includegraphics[width=8.5cm]{Q1Geom_point}}
  %  \vspace{2.0cm}
    \centerline{Question 1: Result 2}\medskip
  \end{minipage}
\end{figure}

This dot plot shows the same information as the line graph before but using a different representation. 
By using dot to exhibit the median house price for each region, the comparison of price between different 
region is quite obvious comapre to the line graph due to overlap of line can be observed in line graph 
which caused diffculty to di the comparison.

Based on the dot plot, it is self-evident that the north east region has houses with the lowest median price 
from 2000 to 2018 while London has the most expensive houses.

\begin{figure}[htb]
  \begin{minipage}[b]{1.0\linewidth}
    \centering
    \centerline{\includegraphics[width=8.5cm]{Q1Geom_gridbar}}
  %  \vspace{2.0cm}
    \centerline{Question 1: Result 3}\medskip
  \end{minipage}
\end{figure}

The last graph for the house price visualisation part is a bar chart which was gained using the same 
median house price data. This bar chart illustrates median house price of 12 regions from 2000 to 2018 
grouped by region. This graph has advantage over the line graph and dot plot when studying the changes of median house price of individual 
regions as it combines the benefit of those two while makes data related to individual region evident.



\subsection{Affordability Visualisation}

\begin{figure}[htb]
  \begin{minipage}[b]{1.0\linewidth}
    \centering
    \centerline{\includegraphics[width=8.5cm]{Q2Geom_line}}
  %  \vspace{2.0cm}
    \centerline{Question 2: Result 1}\medskip
  \end{minipage}
\end{figure}

\begin{figure}[htb]
  \begin{minipage}[b]{1.0\linewidth}
    \centering
    \centerline{\includegraphics[width=8.5cm]{Q2Geom_point}}
  %  \vspace{2.0cm}
    \centerline{Question 2: Result 2}\medskip
  \end{minipage}
\end{figure}

\begin{figure}[htb]
  \begin{minipage}[b]{1.0\linewidth}
    \centering
    \centerline{\includegraphics[width=8.5cm]{Q2Geom_gridbar}}
  %  \vspace{2.0cm}
    \centerline{Question 2: Result 3}\medskip
  \end{minipage}
\end{figure}

\subsection{Impact of Economy Crisis}


\begin{figure}[htb]
  \begin{minipage}[b]{1.0\linewidth}
    \centering
    \centerline{\includegraphics[width=8.5cm]{Q3Geom_gridbar}}
  %  \vspace{2.0cm}
    \centerline{Question 3: Result}\medskip
  \end{minipage}
\end{figure}

\subsection{Correlation Coefficients of House Price}

\begin{figure}[htb]
  \begin{minipage}[b]{1.0\linewidth}
    \centering
    \centerline{\includegraphics[width=8.5cm]{corHeatMap}}
  %  \vspace{2.0cm}
    \centerline{Correlation Coefficients: Result}\medskip
  \end{minipage}
\end{figure}




\section{Evaluation}


% To start a new column (but not a new page) and help balance the last-page
% column length use \vfill\pagebreak.
% -------------------------------------------------------------------------
%\vfill
%\pagebreak

\vfill\pagebreak
\printbibliography


\end{document}
